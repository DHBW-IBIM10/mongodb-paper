\title{\papertitle}

\author{Rocco Schulz and Robert Wawrzyniak}
	
\publishers{Corporate State University\\Baden-Wuerttemberg - Stuttgart}

\date{
\vspace{0.6cm}
provided on 6 May 2013\\
\vspace{0.6cm}
School of: Business\\
\vspace{0.6cm}
Program: International Business Information Management\\
\vspace{0.6cm}
Course: WWI2010I\\
\vspace{0.6cm}
}





% DOCUMENT
\renewcommand{\baselinestretch}{1.5}\normalsize
\begin{document}
\pagestyle{scrheadings}

% roman numerals
\renewcommand{\thepage}{\Roman{page}}
% page numbers centered on top:
\chead{\pagemark}
\cfoot{}

%----------------------------------------------------------------------------
% Title Page
%----------------------------------------------------------------------------

% no page numbering
\thispagestyle{empty}
% include title page
\begin{titlepage}
%\vspace*{\fill}
\begin{center}
\vspace{3mm}

\textbf{\textsc{\Large 	\papertitle}} \\
\vspace{1.5cm}

\papertype \\ % Type of paper (defined in header.tex)
\vspace{0.4cm}

provided on \today \\ % Date of provision
\vspace{0.4cm}

School of: Business  \\
\vspace{0.4cm}

Program: International Business Information Management \\ % Program (defined in header.tex)
\vspace{0.4cm}

Course: WWI2010I \\ % Course (defined in header.tex)
\vspace{1.5cm}

by \\
\paperauthor % Author (defined in header.tex)
\vspace{1cm}
							
Baden-Wuerttemberg Cooperative State University Stuttgart	\\


% Note of confidentiality is only included when `confidential` is set
% to `true` in header.tex
%\textbf{Confidential}\\%
%The content of the paper must not be made available to third parties without approval of the training company.\\


\end{center}
\vspace*{\fill}
\end{titlepage}


%  \begin{abstract}
%  \vspace{1.6cm}
% % \textbf{\abstractname}: 
%  \end{abstract}


%----------------------------------------------------------------------------
% Table of Contents
%----------------------------------------------------------------------------
\renewcommand{\baselinestretch}{1.4}\normalsize
\tableofcontents
\renewcommand{\baselinestretch}{1.5}\normalsize
\newpage


%----------------------------------------------------------------------------
% Abbreviations
%----------------------------------------------------------------------------
% List needs to be indexed after each change.
% This is done by executing the following command:
% ~$ makeindex [filename].nlo -s nomencl.ist -o [filename].nls
\printnomenclature
\addcontentsline{toc}{section}{List of Abbreviations}
\nomenclature{CRUD}{Create Read Update Delete}
\nomenclature{SQL}{Structured Query Language}
\nomenclature{JSON}{JavaScript Object Notation}
\nomenclature{RDBMS}{Relational Database Management System}


\newpage

%----------------------------------------------------------------------------
% List Of Tables
%----------------------------------------------------------------------------
\listoftables
\addcontentsline{toc}{section}{\listtablename}
\newpage


%----------------------------------------------------------------------------
% List of Listings
%----------------------------------------------------------------------------
%\lstlistoflistings
\listoflistings
\addcontentsline{toc}{section}{List of Listings}
\newpage

% Arabic numerals for page numbering
\renewcommand{\thepage}{\arabic{page}}

% Set page number to 1: 
\setcounter{page}{1} 


%----------------------------------------------------------------------------
% Intro
%----------------------------------------------------------------------------
\section{Introduction}
\label{sec:introduction}

The recent growth of user-driven content as well as the availability of sensors
are accounting for huge amounts of growing data which are becoming increasingly
less structured. Handling this semi-structured data has introduced new
challenges for classical processing tools such as relational database management
systems because upfront schema definitions are no longer practical in this
environment.\footcite[Cf.][XVII]{Tiwari_2011} A number of data stores have
evolved that address specific issues related to the processing of big data.
These storage systems, commonly subsumed under the term NoSQL have become
increasingly popular in web applications that need to operate reliably under
extreme load in near real-time.
These usage scenarios have influenced other areas so that new
software patterns were developed resulting in new software products suited
for these types of situations.\footcite[Cf.][VII]{Warden_2011} Many of these products are
open source and freely available for anyone to use.
MongoDB, which is a document-oriented database, is one of these open source
applications that has become very popular over the past years.
A reason for the high popularity is the availability of cheap hardware, which
allows access to large scale data processing for a broad audience.
High prices for professional data software are a strong motivator to turn to
open source alternatives such as MongoDB.\footcite[Cf.][VII]{Warden_2011}


\subsection{Objectives}
\label{sec:objectives}

This paper should familiarize the reader with the basic usage of the
document-oriented database MongoDB. Create, read, update and delete (CRUD)
operations as well as the initial setup of MongoDB should be covered.\\
Additionally deployment-related topics such as configuration, security concerns
and monitoring should be covered to allow a first evaluation for production
usage.\\
This paper does not claim to be a complete guide to MongoDB but rather tries to
provide insights into MongoDB's capabilities.

\subsection{Methodology and Structure}
\label{sec:methodology}

Section \ref{sec:foundation} first provides a brief introduction to the bigger
context of NoSQL databases before explaining the concepts and exemplary use
cases of MongoDB as one representative of NoSQL databases.\\
The following sections focus on the usage of MongoDB. Section \ref{sec:setup}
covers the installation and basic configuration of MongoDB and
\autoref{sec:usage} shows how to run basic create, read, update and delete
operations (CRUD) on MongoDB in different environments such as the mongo shell,
with graphical tools or programmatically via a driver.
Most examples are based on a publicly available data set consisting of several
hundred thousand email messages to allow reproducability of exemplary
results.\\
Section \ref{sec:maintenance} completes the introduction to the topic by
providing essential information on monitoring and other maintenance related
concerns. Section \ref{sec:conclusion} finally concludes and gives a thorough
outlook on advanced topics and future developments of MongoDB.


%----------------------------------------------------------------------------
% Theoretical Foundation
%----------------------------------------------------------------------------
\newpage

\section{Foundation}
\label{sec:foundation}



\subsection{NoSQL Databases}
\label{sec:nosql}

NoSQL describes a range of data-store technologies that do not have an SQL
interface. There are multiple types of NoSQL data stores, these are briefly
introduced in the following sections.

\paragraph{Key-Value Stores}
\label{sec:nosql-key-value}
As the name implies a key-value store maps keys to values in a similar way as a
map or hash-table would in a high level programming
language.\footcite[Cf.][4]{Redmond_2012} Such data structures are very efficient
with an average algorithm running time of big O(1) for accessing data and
are hence often used for caching of data.\footcite[Cf.][p. 14 et sq.]{Tiwari_2011}
However they are not very useful when it is necessary to do queries that are
more complex than simply looking up a value for a given key.\\
There are various open source projects available, whereas memcached, Voldemort,
Redis and Riak belong to the more popular ones.\footcite[Cf.][4]{Redmond_2012}

 
\paragraph{Document Oriented Databases}
\label{sec:nosql-document}
Document oriented databases store documents which can contain a variety of
values which may be of any type, including sub-documents, so that nested 
structures can be created.\footcite[Cf.][p. 5 et sq.]{Redmond_2012}\\
These databases should not be confused with document management systems. Even
though binary documents can be stored as well, the word document in this context
refers to a structured set of keys and values that belong to a
document.\footcite[Cf.][p. 18 et sq.]{Tiwari_2011}.\\
Multiple documents form collections which can be queried. Document oriented
databases do not have a fixed schema and hence exhibit high flexibility.

Document oriented databases are not a new phenomenon as they have been around
for a few decades already. However recently a number of new open source document 
oriented databases like MongoDB and CouchDB have quickly become popular.\footcite[Cf.][p. 18 et sq.]{Tiwari_2011}


\paragraph{Graph Databases}
\label{sec:nosql-graph}
Graph databases are an excellent choice when dealing with highly interconnected
data. They consist of multiple nodes and relationships between these nodes.
Nodes as well as their relationships can contain additional information in the
form of key-value pairs.\footcite[Cf.][6]{Redmond_2012}\\
Graph databases facilitate rapid traversal through graphs by following
relationships. Well known representatives of this group of NoSQL databases are
Neo4j\footnote{See \url{http://www.neo4j.org/}} and FlockDB\footnote{See
\url{https://github.com/twitter/flockdb}}.\footcite[Cf.][19]{Tiwari_2011}


\subsection{MongoDB}
\label{sec:mongodb}


``MongoDB is designed to be huge (the name mongo is extracted from the word
humongous). Mongo server configurations attempt to remain consistent—if
you write something, subsequent reads will receive the same value (until the
next update). This feature makes it attractive to those coming from an RDBMS
background. It also offers atomic read-write operations such as incrementing
a value and deep querying of nested document structures. Using JavaScript
for its query language, MongoDB supports both simple queries and complex
map-reduce jobs.''\footcite[][6]{Redmond_2012}

MongoDB belongs to the group of document oriented NoSQL databases and is written
in C++. It does not try to be a general purpose database but has been developed
with the philosophy ``One size does not fit all'' in mind.\footcite[Cf.][3]{Plugge_2010}

MongoDB differs from other document-oriented databases in its ability to easily
scale across several servers. This is possible with replication, where data is 
copied to other servers to provide redundancy or by sharding where large
collections are split into pieces to be distributed among all
nodes.\footcite[Cf.][165]{Redmond_2012}

\paragraph{Terminology}
The MongoDB terminology differs from the terms used in RDBMS due to the
different concepts. MongoDB works with documents instead of tables as opposed to conventional RDBMS.
A document is a single storage unit and multiple documents can be contained in a
collection. Documents are stored in BSON format which is a JSON based
specification.\footcite[Cf.][]{bson_spec}

Most terms can roughly be mapped to their SQL counterparts. An overview of key
terms is provided in Tab. \ref{tab:sql-mongo-map}.

 %allows proper citation marks inside tables
\begin{table}[ht]
\begin{tabular*}{0.95\textwidth}{p{0.45\textwidth} p{0.45\textwidth}}
\toprule
\textbf{SQL Terms/Concepts} & \textbf{MongoDB Terms/Concept}\\
\midrule 
database & 	database\\
table 	 &	collection\\
row 	 &	document\\
column 	 &	field\\
index 	 &	index\\
table joins &	embedded documents and linking\\
primary key\newline Specify any unique column or column combination as primary
key.
		&
primary key\newline
In MongoDB, the primary key is automatically set to the \_id field.\\

\bottomrule 
\end{tabular*}
  \begin{savenotes}
  \caption[SQL terms mapped to MongoDB]{SQL terms mapped to
  MongoDB\footcite[][]{mongo_sql_comp} }
  \label{tab:sql-mongo-map}
  \end{savenotes}
\end{table}


\paragraph{Areas of application}
A design goal of MongoDB is high availability through
replication.\footcite[Cf.][3]{Plugge_2010}
MongoDB aims for good horizontal scalability, speed and simplicity. Transaction
support has been omitted as a trade-off in favor of these
attributes.\footcite[Cf.][5]{Plugge_2010}\\
MongoDB is hence a good fit for applications that require a fast and highly
available data store, which is one of the reasons why it is very popular for web
applications. Due to its document-oriented nature MongoDB is a good fit for
hierarchical data such as category hierarchies and comments.\\
In MongoDB comments can be contained in their parent document as a
sub-collection. Support for threaded comments can be added by embedding comments
in their parent comment. As a result the whole comment thread can be obtained
 by simply retrieving the embedded collection of comments from an item.
In a relational database this could only be achieved by more sophisticated
queries that require joins.

As opposed to RDBMS, MongoDB does not have fixed formats for the items that
are to be stored. Differed types of objects with different sets of attributes
can easily be put into the same collection. \\
E-Commerce product catalogs must have this capability in order to store
information of various articles. In relational databases this capability could be provided
via multiple table inheritance.
While this allows flexible structures, it also requires expensive join operations 
to retrieve all relevant attributes of a product.
``MongoDB’s dynamic schema means that each document need not conform to the same
schema. As a result, the document for each product only needs to contain
attributes relevant to that product.''\footcite[][]{mongo_product_catalog}

    
\FloatBarrier
%----------------------------------------------------------------------------
% Hands on Mongo
%----------------------------------------------------------------------------
\clearpage
\section{Setting up MongoDB}
\label{sec:setup}

\subsection{Installation}
\label{sec:installation}

\paragraph{Windows}
\label{sec:installation-windows}
The latest version of MongoDB can be downloaded from \url{http://www.mongodb.org/downloads}.
After downloading, the archive needs to be extracted, e.g. to \texttt{C:\textbackslash \-mongo\-db-win32-x86\-\_64-[version]}.
Afterwards the environment needs to be set up by running the following sequence of commands in the
Windows Command Prompt.

\begin{listing}
    \begin{batcode}
cd \
:: rename folder
move C:\mongodb-win32-* C:\mongodb
:: create folder for data in default location
md data
md data\db
    \end{batcode}
    \caption{Commands to set up the MongoDB environment on Windows}
    \label{lst:win-setup}
\end{listing}

MongoDB can now be started by issuing the command \texttt{C:\textbackslash mongodb\textbackslash\- bin\textbackslash\- mongod.exe}.
Since version 2 it is also possible to run mongod as a service on Windows.\footcite[Cf.][]{mongo_install_win} Refer
to the online documentation for instructions on how to do 
this.\footnote{See \url{http://docs.mongodb.org/manual/tutorial/install-mongodb-on-windows/}}

\paragraph{Linux}
\label{sec:installation-linux}
On Linux MongoDB can be installed via the integrated package manager.
Most distributions do not ship the latest version of MongoDB. This paper describes the usage of
MongoDB version 2.4 which can be obtained from 10gen's own repository if not available in the
Linux distribution. See \url{http://docs.mongodb.org/manual/installation/}

\subsection{Configuration}
\label{sec:configuration}

There are two configuration interfaces for MongoDB: The command line and the
configuration file.
When installed on Linux, MongoDB comes with a default configuration file under
\texttt{/etc/mongodb.conf} which is shown in appendix \ref{apx:mongo-conf} for
convenience.\footcite[Cf.][]{mongo_config} This file will be used automatically
when mongod is run as a service, but needs to be specified explicitly when
starting mongod via the \texttt{mongod} command (\texttt{mongod -f
/etc/mongod.conf}).\\
In order to start MongoDB with a different data storage path on port $20000$,
the following command could be issued:  \texttt{mongod -dbpath /media/mongodisk/data
-port 20000}.
Additional options are available as documented in the manual.\footnote{See
\url{http://docs.mongodb.org/manual/administration/configuration/}} Section
\ref{sec:maintenance-security} also shows some security related configurations.

    
\newpage
\section{Using MongoDB}
\label{sec:usage}

\subsection{Test Environment Setup}
\label{sec:test-environment}

The following usage scenarios will be based on a publicly available data-set.
This section describes how to obtain and load this data-set into the previously
installed MongoDB.

The Enron Email Corpus has been ported to MongoDB and can be obtained from
\url{http://mongodb-enron-email.s3-website-us-east-1.amazonaws.com/}.
The uncompressed data-set contains approximately 500,000 emails.
While this is a relatively small data-set it is sufficient for
testing basic features of the database.

Once the archive has been downloaded and unpacked it can be loaded into MongoDB
by using the command line tool
\texttt{mongorestore}.\footcite[Cf.][]{mongo_restore}
 On Linux, run the commands as shown in Lst. \ref{lst:load-enron}.

\begin{listing}
    \begin{bashcode}
    wget https://s3.amazonaws.com/mongodb-enron-email/enron_mongo.tar.bz2
    tar -xjvf enron_mongo.tar.bz2
    mongorestore --db enron --noobjcheck dump/enron_mail/
    \end{bashcode}
    \caption{Shell commands to download and load the enron email corpus into MongoDB on Linux}
    \label{lst:load-enron}
\end{listing}

Note that the loading process requires a running instance of the MongoDB server.
Verify that the import was successful by opening the mongo shell and typing
\texttt{show dbs}.
The output should contain a row with \texttt{enron 3.9521484375GB}\footnote{The
size of the database may differ due to different file systems}, which is the
database that has been created during the loading process.


\subsection{Command Line Interface}
\label{sec:usage-cli}

The command line interface for MongoDB is called the mongo shell and is the most powerful
tool for operating on MongoDB databases. The shell language is JavaScript, so that own
functions and variables can be declared within the shell.
The following examples assume that the Enron email corpus has been loaded into the database \textit{enron}.
In order to start working on this database, type \texttt{use enron}.

\subsubsection{Read}
Before performing any read operations it is necessary to figure out what collections
there are in the enron database, that can be read from. This information can be accessed
via the command \texttt{show collections} as shown in Lst. \ref{lst:show-collections}

\begin{listing}
    \begin{javascriptcode}
> show collections
messages
system.indexes
    \end{javascriptcode}
    \caption{Command to get a list of available collections in the current database}
    \label{lst:show-collections}
\end{listing}

\paragraph{Making Queries}
In the mongo shell, the \texttt{find()} and \texttt{findOne()} functions can be used to
query for data.\footcite[Cf.][7]{mongo_crud_manual}
These queries are performed by calling them on any collection.
Obtaining all emails from the messages collection is e.g. done by running
\texttt{db.messages.find()}.

The syntax of the find command is as follows: \texttt{db.collection.find(
<query>, <projection> )}\\
The \texttt{<query>} argument is used for filtering and the \texttt{<projection>} is used to specify
or limit the fields that should be returned.
The projection is hence comparable to the \textit{SELECT} part of a SQL query and the
query parameter is comparable to the \textit{WHERE} clause of a SQL query.


The \texttt{findOne()} function is very similar, however it only returns the first
element that matches the query. Lst. \ref{lst:findone} shows the result of the
\texttt{findOne} query on the messages collection without a query document provided. The
second argument tells MongoDB to exclude the \textit{body} field from the result. Setting
the projection object to \texttt{\{"subject":1\}} would tell MongoDB to exclude
all fields from the result except for \textit{body}.

\begin{listing}
    \begin{javascriptcode}
> db.messages.findOne({}, {"body": 0})
{
    "_id" : ObjectId("4f16fc97d1e2d32371003e27"),
    "subFolder" : "notes_inbox",
    "mailbox" : "bass-e",
    "filename" : "450.",
    "headers" : {
        "X-cc" : "",
        "From" : "michael.simmons@enron.com",
        "Subject" : "Re: Plays and other information",
        "X-Folder" : "\\Eric_Bass_Dec2000\\Notes Folders\\Notes inbox",
        "Content-Transfer-Encoding" : "7bit",
        "X-bcc" : "",
        "To" : "eric.bass@enron.com",
        "X-Origin" : "Bass-E",
        "X-FileName" : "ebass.nsf",
        "X-From" : "Michael Simmons",
        "Date" : "Tue, 14 Nov 2000 08:22:00 -0800 (PST)",
        "X-To" : "Eric Bass",
        "Message-ID" : "<6884142.1075854677416.JavaMail.evans@thyme>",
        "Content-Type" : "text/plain; charset=us-ascii",
        "Mime-Version" : "1.0"
    }
}
    \end{javascriptcode}
    \caption{\texttt{findOne} query on the messages collection}
    \label{lst:findone}
\end{listing}

As can be seen from the output in Lst. \ref{lst:findone} there can also be nested documents
within one document. In this case the headers field contains an object with all header fields.
MongoDB allows querying for nested attributes as well. So in order to find out the percentage of
replies among all emails we could run a series of commands as shown in Lst. \ref{lst:query-percent-replies}.

\begin{listing}
    \begin{javascriptcode}
> nAll = db.messages.count()
501513
> nReply = db.messages.find({"headers.Subject": /Re:.*/}).count()
100195
> nReply / nAll * 100
19.978544923062813
    \end{javascriptcode}
    \caption[Calculating the percentage of replies]{Calculating the percentage of replies in the messages collection within the mongo shell}
    \label{lst:query-percent-replies}
\end{listing}

Similarly it is possible to query for values which are stored in an array
instead of a sub-document.
If the array contains sub-documents it is also possible to query for specific
fields within the sub-documents.\footcite[Cf.][10]{mongo_crud_manual} This is
demonstrated in Lst. \ref{lst:array-query}

\begin{listing}
    \begin{javascriptcode}
//find documents with exactly this array in the tags attribute
db.inventory.find( { tags: ['fruit','food','citrus'] } )
//find documents that contain a 'fruit' element in their tags array
db.inventory.find( { tags: 'fruit'} )
//find documents that have 'fruit' as the first element in the tags array
db.inventory.find( {'tags.0': 'fruit'} )
//find documents where the memos field contains an array that contains at least one
//subdocument with the field 'by' with the value 'shipping'
db.inventory.find( {'memos.by': 'shipping'} )
    \end{javascriptcode}
    \caption{Array queries on an inventory collection}
    \label{lst:array-query}
\end{listing}

Besides using regular expressions as demonstrated in Lst. \ref{lst:query-percent-replies} MongoDB
offers a number of other query selectors for comparisons (see Tab. \ref{tab:query-selectors-compare}).

An exemplary query using the greater than (\$gt), less than (\$lt) and or (\$or)
operator is shown in Lst. \ref{lst:query-operators}.
This query selects all documents in the inventory collection which have a type attribute with
a value of 'food' and either have  a quantity attribute with a value above $100$ or a price attribute
with a value below $9.95$.

\begin{listing}
    \begin{javascriptcode}
db.inventory.find( { type:'food', $or:[ { qty:{ $gt:100} },{ price:{ $lt:9.95} } ]} )
    \end{javascriptcode}
    \caption[Exemplary usage of query operators]{Exemplary usage of query operators\footcite[][9]{mongo_crud_manual}}
    \label{lst:query-operators}
\end{listing}


\begin{savenotes}
\begin{table}[htbp]
\begin{tabular*}{\textwidth}{p{0.1\textwidth} p{0.85\textwidth}}
\toprule
\textbf{Name} 					& \textbf{Description}\\
\midrule 
\$all    & Matches arrays that contain all elements specified in the query. \\
\$gt     & Matches values that are greater than the value specified in the query.\\
\$gte    & Matches values that are equal to or greater than the value specified in the query.\\
\$in     & Matches any of the values that exist in an array specified in the query.\\
\$lt     & Matches vales that are less than the value specified in the query.\\
\$lte    & Matches values that are less than or equal to the value specified in the query.\\
\$ne     & Matches all values that are not equal to the value specified in the query.\\
\$nin    & Matches values that do not exist in an array specified to the query.\\
\bottomrule 
\end{tabular*}
  \caption[Query selectors for value comparison]{Query selectors for value comparison\footcite[][]{mongo_query_ops}}
  \label{tab:query-selectors-compare}
\end{table}
\end{savenotes}


\paragraph{Optimizing queries with indices}
When sorting a data-set MongoDB inspects all documents within the collection
to put them in the desired order. In large collections this can be very time
consuming and results in long response times.
MongoDB actually refuses to run queries which would take too long to complete as shown in
Lst. \ref{lst:no-index}.

\begin{listing}
    \begin{javascriptcode}
//find all emails and sort them ascending by the senders email address
> db.messages.find().sort({"headers.From":1})
error: {
    "$err" : "too much data for sort() with no index.  add an index or specify a smaller limit",
    "code" : 10128
}
    \end{javascriptcode}
    \caption{Trying to sort the messages collection by a non-indexed field}
    \label{lst:no-index}
\end{listing}

Just like in other databases indices can be created to improve the efficieny of
read operations.\footcite[Cf.][152 et sq.]{Redmond_2012} Indices can be created
using the \texttt{ensureIndex({<field1>: <order>, <field2>: <order>,... })}
function in the mongo shell.\footcite[Cf.][12]{mongo_crud_manual} 
The internal mechanisms behind indexing of MongoDB are beyond the scope of this
paper and can be read about in the official documentation.\footnote{See
\url{http://docs.mongodb.org/manual/core/indexes/}} An index for the
\textit{From} field in the embedded \textit{headers} document can be created
with \texttt{db.messages.ensureIndex({"headers.From":1})}, so that the query
shown in Lst. \ref{lst:no-index} can be executed. It is possible to display all
current indices by executing \texttt{getIndices()} as shown in Lst.
\ref{lst:showIndices}.

\begin{listing}
    \begin{javascriptcode}
> db.messages.getIndices()
[   {
        "v" : 1,
        "key" : { "_id" : 1 },
        "ns" : "enron.messages",
        "name" : "_id_"
    },
    {
        "v" : 1,
        "key" : { "headers.From" : 1 },
        "ns" : "enron.messages",
        "name" : "headers.From_1"
    } ]
    \end{javascriptcode}
    \caption{Showing all indices of the messages collection}
    \label{lst:showIndices}
\end{listing}

Further optimization strategies include query tuning, storing pre-calculated
data and restructuring document formats and are described in detail in
\citetitle{Chodorow_2011_tips} by
\citeauthor{Chodorow_2011_tips}.\footcite[Cf.][]{Chodorow_2011_tips}

\FloatBarrier
\subsubsection{Create}
Documents can be created with the \texttt{insert()} command in the mongo
shell.\\
Bulk inserts are also possible (see listing \ref{lst:insertDocuments}).\footcite[Cf.][p. 53 et sqq.]{mongo_crud_manual}

\begin{code}
    \javascriptfile{lst/insert.js}
    \caption{(Bulk) Insertion of Documents}
    \label{lst:insertDocuments}
\end{code}

\FloatBarrier
\pagebreak

\subsubsection{Update}
Updates can be done with the \texttt{update()} command and the \texttt{save()}
command (see listing \ref{lst:updateDocuments}).\footcite[Cf.][p. 75 et. sqq.]{mongo_crud_manual}

\begin{code}
	\javascriptfile{lst/update.js}
	\caption{Updating MongoDB Collections}
	\label{lst:updateDocuments}
\end{code}

\begin{savenotes}
\begin{table}[htbp]
\begin{tabular*}{\textwidth}{p{0.2\textwidth} p{0.75\textwidth}}
\toprule
	\textbf{Name} 					& \textbf{Description}\\
	\midrule 
	\$inc			& Increments the value of the field by the specified amount.\\
	\$rename		& Renames a field.\\
	\$setOnInsert	& Sets the value of a field upon documentation creation during an upsert. Has no effect on update operations that modify existing documents.\\
	\$set			& Sets the value of a field in an existing document.\\
	\$unset			& Removes the specified field from an existing document.\\
\bottomrule 
\end{tabular*}
  \caption[]{Field Update Operators\footcite[][]{mongo_update}}
  \label{tab:query-selectors-compare}
\end{table}
\end{savenotes}

\begin{savenotes}
\begin{table}[htbp]
\begin{tabular*}{\textwidth}{p{0.2\textwidth} p{0.75\textwidth}}
\toprule
	\textbf{Name} 					& \textbf{Description}\\
	\midrule 
	\midrule
	\textbf{Operators} & \\
	\midrule 
	\midrule
	\$			&	Acts as a placeholder to update the first element that matches the query condition in an update.\\
	\$addToSet	&	Adds elements to an existing array only if they do not already exist in the set.\\
	\$pop		&	Removes the first or last item of an array.\\
	\$pullAll	&	Removes multiple values from an array.\\
	\$pull		&	Removes items from an array that match a query statement.\\
	\$pushAll	&	Deprecated. Adds several items to an array.\\
	\$push		&	Adds an item to an array.\\
	\midrule
	\midrule	
	\textbf{Modifiers} & \\
	\midrule
	\midrule
	\$each		&	Modifies the \$push and \$addToSet operators to append multiple items for array updates.\\
	\$slice		&	Modifies the \$push operator to limit the size of updated arrays.\\
	\$sort		&	Modifies the \$push operator to reorder documents stored in an array.\\	
\bottomrule 
\end{tabular*}
  \caption[]{Array Update Operators\footcite[][]{mongo_update}}
  \label{tab:query-selectors-compare}
\end{table}
\end{savenotes}

\FloatBarrier

\subsubsection{Delete}
Delete operations for one or multiple documents can be performed with
\texttt{db.collection.remove()} (see listing \ref{lst:removeDocuments}).\footcite[Cf.][p. 83 et.
sqq.]{mongo_crud_manual}

\begin{code}
	\javascriptfile{lst/remove.js}
	\caption{Removing Documents from a MongoDB Collection}
	\label{lst:removeDocuments}
\end{code}

\FloatBarrier

\subsection{Programmatically}
\label{sec:usage-programmatically}
In order to access MongoDB programmatically it is necessary to use a
\textit{driver} to connect to the database and to translate MongoDB constructs
into appropriate ones for your target language. 10gen provides drivers for most
languages (with additional ones developed by the community) including but not
limited to:\footcite[Cf.][]{mongo_drivers}
\begin{enumerate}
\item C/++/\#
\item Java
\item JavaScript/Node.js
\item PHP
\item Python
\item Ruby
\end{enumerate}

This section will show how to use MongoDB programmatically using the Ruby language.

\paragraph{Installing the driver}
\begin{code}
	\bashfile{lst/install_mongo_gem.sh}
	\caption{Installing the mongo gem}
	\label{lst:installMongoGem}
\end{code}

The driver allows interacting with MongoDB through Ruby by employing similar
constructs to the examples above using the Mongo shell, however most developers
use an Object Document Mapper (ODM) to gain higher-level abstractions and
modeling functionality.\footcite[Cf.][]{mongo_ruby_driver} This paper will focus
on Mongoid, one of the most popular ODMs.

\paragraph{Installing Mongoid}
\begin{code}
	\bashfile{lst/install_mongoid.sh}
	\caption{Installing the Mongoid gem}
	\label{lst:installMongoid}
\end{code}

After installing the gem, it is necessary to configure Mongoid so that it
connects to the correct MongoDB instance. Mongoid is configured by creating an
modifying the file \texttt{mongoid.yml}, of which an example is located in
appendix \ref{apx:mongoid}.

The exact mechanism of loading the configuration is dependent upon which
framework (if any) is used by the developer. Proper guidance as to how to
proceed can be found in the Mongoid documentation, this includes advanced
features such as logging, sharding and replica
sets.\footcite[Cf.][]{mongoid_install}

\paragraph{Using Mongoid}
Listing \ref{lst:mongoid} shows essential usage of the Mongoid ODM. In addition
to the highlighted features, Mongoid allows idiomatic usage of every MongoDB
feature.
Thus the resulting feature set and documentation is quite extensive and is
recommended to gain a deeper understanding, see
\url{http://mongoid.org/en/mongoid/} for more details.

\begin{code}
	\rubyfile{lst/mongoid.rb}
	\caption{Exemplary Usage of Mongoid}
	\label{lst:mongoid}
\end{code}


\newpage
\section{Maintaining MongoDB}
\label{sec:maintenance}

\subsection{Monitoring}
\label{sec:maintenance-monitroing}
Monitoring is an elementary part of database administration in order to avoid
incidents before they happen. MongoDB ships with separate tools for monitoring.
One of these is \texttt{mongotop} which can be used to get performance information
on collections level.\footcite[Cf.][]{mongo_monitoring}
\begin{listing}
    \begin{bashcode}
                            ns       total        read       write
2013-06-25T19:28:48
                enron.messages      1158ms      1158ms         0ms
          enron.system.indexes         0ms         0ms         0ms
       enron.system.namespaces         0ms         0ms         0ms
            enron.system.users         0ms         0ms         0ms
             local.startup_log         0ms         0ms         0ms
          local.system.indexes         0ms         0ms         0ms
    \end{bashcode}
    \caption{Output of mongotop}
    \label{lst:mongotop}
\end{listing}

Another command line tool for monitoring is \texttt{mongostat}, which provides
information on a higher level such as number of open connections,
memory consumption, network traffic and query throughput.\footcite[Cf.][]{mongo_monitoring}

Additionally it is possible to get these information via a web interface.
``In default configurations the REST interface is accessible on 28017. For
example, to access the REST interface on a locally running mongod instance:
\url{http://localhost:28017}''\footcite[][]{mongo_monitoring}.
The REST interface is disabled by default and needs to be enabled by setting the
\texttt{rest} field to \texttt{true} in the configuration file.\footcite[Cf.][]{mongo_conf}\\
It is also possible to get these information from within the mongo shell via commands such as
\texttt{db.serverStatus()}. Additionally profiling information can be obtained when the
profiling is enabled. This can be done by setting the profiling level to $1$ or $2$ via
\texttt{db.setProfilingLevel(<level>)}. When set to $1$, all operations are profiled, when set
to $2$ only slow operations will be profiled.
A full list of available commands is available in the online
documentation.\footnote{See \url{http://docs.mongodb.org/manual/reference/command/\#diagnostic-commands}}


\subsection{Security and Authentication}
\label{sec:maintenance-security}
It is recommended to run MongoDB in a trusted environment, to make sure that
only trusted machines may access the MongoDB
server.\footcite[Cf.][118]{Chodorow_2010}
If MongoDB is run on a server which is publicly accessible, it should be
run with the \texttt{--bindip} option to specify a local IP address that MongoDB
will be bound to (e.g. \texttt{--bindip localhost} to only allow
connections from an application on the same
machine).\footcite[Cf.][118]{Chodorow_2010}

Access restrictions can be put in place by using MongoDB's authentication system.
Before using the authentication framework one should
create an administration user with the commands shown in Lst.
\ref{lst:mongo-root}.\footcite[Cf.][]{mongo_adminuser} Note that the creation of
an administration user is only possible when working on localhost with MongoDB's localhost authentication
bypass enabled, which allows full administrative access to local users.
This setting should be disabled after the creation of an administration user via
\texttt{mongod --setParameter enableLocalhostAuthBypass=0} during startup or by
setting this option in the configuration file.

\begin{listing}
    \begin{javascriptcode}
//switch to the admin DB
> db = db.getSiblingDB('admin')
admin
// create a superuser with the role "userAdminAnyDatabase"
> db.addUser({user:"root",pwd:"toor",roles: ["userAdminAnyDatabase"]})
{
	"user" : "root",
	"pwd" : "6a921fa21bbcd22989efecbcb2340d17",
	"roles" : ["userAdminAnyDatabase"],
	"_id" : ObjectId("51d16db53ac63115056d6341")
}
    \end{javascriptcode}
    \caption{Creating an administration user}
    \label{lst:mongo-root}
\end{listing}

Users of a database are stored as documents in the \textit{system.users} collection
within this database.
User documents consist of a \textit{user} field which contains the username,
a \textit{readOnly} flag and a \textit{pwd} field which stores a hash of the
username and password.\footcite[Cf.][]{mongo_adduser}\\
Authentication in MongoDB can be enabled by either starting \texttt{mongod} with
the \texttt{--auth} switch or by configuring it in the settings file with the
line \texttt{auth = true}.

Lst. \ref{lst:add-user}
shows how a user can be added to the enron database and how this user
can be used to authenticate in the shell.

\begin{listing}
    \begin{javascriptcode}
> use enron
switched to db enron
> show collections
Mon Jul  1 14:11:13.055 JavaScript execution failed: error: {
	"$err" : "not authorized for query on enron.system.namespaces",
	"code" : 16550
} at src/mongo/shell/query.js:L128
//add a user with read and write access
> db.addUser({user:"mike", pwd:"abc123", roles: ["readWrite"]})
{
	"user" : "mike",
	"pwd" : "fbfddbdf3a7a9955a764d02b4c10fd84",
	"roles" : ["readWrite"],
	"_id" : ObjectId("51d169773ac63115056d6340")
}
//try to authenticate with wrong pw
> db.auth({user:"mike",pwd:"test"})
Error: 18 { code: 18, ok: 0.0, errmsg: "auth fails" }
0
> db.auth({user:"mike",pwd:"abc123"})
1
> show collections
messages
system.indexes
system.users
    \end{javascriptcode}
    \caption[Adding a user to the enron database]{Adding a user to the enron
    database with the \texttt{auth} option set to \texttt{true}}
    \label{lst:add-user}
\end{listing}




%----------------------------------------------------------------------------
% Closing
%----------------------------------------------------------------------------

\clearpage
\section{Conclusion and Outlook}
\label{sec:conclusion}

MongoDB's strengths are its ability to handle large amounts of data as well as
large amounts of requests by replication and sharding. Additionally it allows
very flexible data models due to its schema-less storage model.\\
In \autoref{sec:usage} it has been shown that MongoDB is easy to use as its
query language shares many similarities with traditional SQL concepts.\\
MongoDB should not be seen as an alternative for conventional RDBMS but as a
complementary module in application development. MongoDB encourages
denormalization of schemas by not having any. This goes along with less
constraints so that more caution is needed on the application development side.

There are various other features available in MongoDB which haven't been covered
in this paper. These include auto-sharding, geospatial lookups, advanced
optimization and clustering.\\
Designing collections or document structures for MongoDB is also different to
the design of relational database models and should be further looked into
before considering the usage of MongoDB in production.



%----------------------------------------------------------------------------
% APPENDIX
%----------------------------------------------------------------------------
% Appendix sections need to be within the subappendices environment.
% Use the command \appsection{title} instead of \section to introduce each
% appendix. This will add each appendix to the list of appendices.

% sets the appendix environment and resets the section counters
\newpage \begin{appendices} 
\appendixtocon %adds an 'Appendices' entry to the toc

\appendixpage %prints the title on the page

\subsection*{\listappendixname}
%--------------------------------
% style of the \listofappendices command is defined in header.tex
\listofappendices

% begin appendices on a new page
\newpage

%start environment for subappendices, so that new sections are formatted as
%subsections of appendix
\begin{subappendices}
\renewcommand{\setthesubsection}{\arabic{subsection}:}%

\appsection{MongoDB Configuration File}
\label{apx:mongo-conf}

\begin{code}
\inifile{lst/mongod.conf}
\caption{Sample MongoDB configuration file}
\end{code}

\newpage
\appsection{Overview of the Mongoid Configuration File 'mongoid.yml'}
\label{apx:mongoid}
\begin{code}
	\rubyfile{lst/mongoid.yml}
	\caption{Example mongoid.yml}
\end{code}

% close the appendices environment
\end{subappendices}
\end{appendices}
