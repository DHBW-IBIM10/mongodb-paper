\title{\papertitle}

\author{Rocco Schulz and Robert Wawrzyniak}
		
\publishers{Corporate State University\\Baden-Wuerttemberg - Stuttgart}

\date{
\vspace{0.6cm}
provided on 6 May 2013\\
\vspace{0.6cm}
School of: Business\\
\vspace{0.6cm}
Program: International Business Information Management\\
\vspace{0.6cm}
Course: WWI2010I\\
\vspace{0.6cm}
}





% DOCUMENT
\renewcommand{\baselinestretch}{1.5}\normalsize
\begin{document}
\pagestyle{scrheadings}

% roman numerals
\renewcommand{\thepage}{\Roman{page}}
% page numbers centered on top:
\chead{\pagemark}
\cfoot{}

%----------------------------------------------------------------------------
% Title Page
%----------------------------------------------------------------------------

% no page numbering
\thispagestyle{empty}
% include title page
\begin{titlepage}
%\vspace*{\fill}
\begin{center}
\vspace{3mm}

\textbf{\textsc{\Large 	\papertitle}} \\
\vspace{1.5cm}

\papertype \\ % Type of paper (defined in header.tex)
\vspace{0.4cm}

provided on \today \\ % Date of provision
\vspace{0.4cm}

School of: Business  \\
\vspace{0.4cm}

Program: International Business Information Management \\ % Program (defined in header.tex)
\vspace{0.4cm}

Course: WWI2010I \\ % Course (defined in header.tex)
\vspace{1.5cm}

by \\
\paperauthor % Author (defined in header.tex)
\vspace{1cm}
							
Baden-Wuerttemberg Cooperative State University Stuttgart	\\


% Note of confidentiality is only included when `confidential` is set
% to `true` in header.tex
%\textbf{Confidential}\\%
%The content of the paper must not be made available to third parties without approval of the training company.\\


\end{center}
\vspace*{\fill}
\end{titlepage}


%  \begin{abstract}
%  \vspace{1.6cm}
% % \textbf{\abstractname}: 
%  \end{abstract}


%----------------------------------------------------------------------------
% Table of Contents
%----------------------------------------------------------------------------
\tableofcontents
\newpage


%----------------------------------------------------------------------------
% Abbreviations
%----------------------------------------------------------------------------
% List needs to be indexed after each change.
% This is done by executing the following command:
% ~$ makeindex [filename].nlo -s nomencl.ist -o [filename].nls
\printnomenclature
\addcontentsline{toc}{section}{List of Abbreviations}
\nomenclature{CRUD}{Create Read Update Delete}
\nomenclature{SQL}{Structured Query Language}
\nomenclature{JSON}{JavaScript Object Notation}
\nomenclature{RDBMS}{Relational Database Management System}


\newpage

%----------------------------------------------------------------------------
% List Of Tables
%----------------------------------------------------------------------------
\listoftables
\addcontentsline{toc}{section}{\listtablename}
\newpage


%----------------------------------------------------------------------------
% List of Figures
%----------------------------------------------------------------------------
\listoffigures
\addcontentsline{toc}{section}{\listfigurename}
\newpage

%----------------------------------------------------------------------------
% List of Listings
%----------------------------------------------------------------------------
%\lstlistoflistings
\listoflistings
\addcontentsline{toc}{section}{List of Listings}
\newpage

% Arabic numerals for page numbering
\renewcommand{\thepage}{\arabic{page}}

% Set page number to 1: 
\setcounter{page}{1} 


%----------------------------------------------------------------------------
% Intro
%----------------------------------------------------------------------------
\section{Introduction}
\label{sec:introduction}

Introduction to NoSQL.

% These\footcite[Cf.][22]{Warden_2011} are\footcite[Cf.][23]{Tiwari_2011} just some
% dummy\footcite[Cf.][34]{Redmond_2012} references\footcite[][]{Chodorow_2011_tips}
% to test\footcite[Cf.][44]{Chodorow_2011_scaling} the bibliography\footcite[Cf.][44]{Ohiggins_2011}
% environement.\footcite[Cf.][]{mongo_manual}


\subsection{Objectives}
\label{sec:objectives}
%SMART - specific, measurable, achievable, relevant, timed

\begin{enumerate}
  \item Test installation of mongoDB
  \item Introduce basic usage (CRUD + simple analytics)
  \item Performance tests with large datasets
  \item Show programmatical usage of mongoDB in a Python web project
\end{enumerate}

\subsection{Methodology and Structure}
\label{sec:methodology}

TBD


%----------------------------------------------------------------------------
% Theoretical Foundation
%----------------------------------------------------------------------------
\newpage

\section{Foundation}
\label{sec:foundation}



\subsection{NoSQL Databases}
\label{sec:nosql}

\begin{enumerate}
  \item nosql describes a range of data store technologies that do not have an sql interface
  \item there are multiple types of nosql data stores, these are briefly introduced in the following sections
\end{enumerate}
Main differences between RDBS and noSQL should be listed to avoid that readers
look at things as if they were conventional databases. they are not.

\paragraph{Key-Value Stores}
\label{sec:nosql-key-value}
As the name implies a key-value store maps keys to values in a similar way as a
map or hashtable would in a high level programming
language.\footcite[Cf.][4]{Redmond_2012} Such data structures are very efficient
with an average algorithm running time of big O(1) for accessing data and
are hence often used for caching of data.\footcite[Cf.][p. 14 et sq.]{Tiwari_2011}
However they are not very useful when it is necessary to do queries that are
more complex than simply looking up a value for a given key.\\
There are various open source projects available, whereas memcached, Voldemort,
Redis and Riak belong to the more popular ones.\footcite[Cf.][4]{Redmond_2012}

 
\paragraph{Document Oriented Databases}
\label{sec:nosql-document}
Document oriented databases store documents which can contain a variety of
values which may be of any type, including sub-documents, so that nested 
structures can be created.\footcite[Cf.][p. 5 et sq.]{Redmond_2012}\\
These databases should not be confused with document management systems. Even
though binary documents can be stored as well, the word document in this context
refers to a structured set of keys and values that belong to a
document.\footcite[Cf.][p. 18 et sq.]{Tiwari_2011}.\\
Multiple documents form collections which can be queried. Document oriented
databases do not have a fixed schema and hence exhibit high flexibility.

Document oriented databases are not a new phenomenom as they have been around
for a few decades already. However recently a number of new open source document 
oriented databases like MongoDB and CouchDB have quickly become popular.\footcite[Cf.][p. 18 et sq.]{Tiwari_2011}


\paragraph{Graph Databases}
\label{sec:nosql-graph}
Graph databases are an excellent choice when dealing with highly interconnected
data. They consist of multiple nodes and relationships between these nodes.
Nodes as well as their relationships can contain additional information in the
form of key-value pairs.\footcite[Cf.][6]{Redmond_2012}\\
Graph databases facilitate rapid traversal through graphs by following
relationships. Well known representatives of this group of noSQL databases are
Neo4j\footnote{See \url{http://www.neo4j.org/}} and FlockDB\footnote{See
\url{https://github.com/twitter/flockdb}}.\footcite[Cf.][19]{Tiwari_2011}


\subsection{MongoDB}
\label{sec:mongodb}


``MongoDB is designed to be huge (the name mongo is extracted from the word
humongous). Mongo server configurations attempt to remain consistent—if
you write something, subsequent reads will receive the same value (until the
next update). This feature makes it attractive to those coming from an RDBMS
background. It also offers atomic read-write operations such as incrementing
a value and deep querying of nested document structures. Using JavaScript
for its query language, MongoDB supports both simple queries and complex
mapreduce jobs.''\footcite[][6]{Redmond_2012}

\begin{enumerate}
  \item  noSQL store that belongs to the group of document stores
  \item Documents are stored as JSON or binary JSON
  \item development history and some general information
  \item use cases
\end{enumerate}    

    
    

%----------------------------------------------------------------------------
% Hands on Mongo
%----------------------------------------------------------------------------
\section{Setting up MongoDB}
\label{sec:setup}

\subsection{Installation}
\label{sec:installation}

\subsubsection{Windows}
\label{sec:installation-windows}
Download and follow install instructions.

\subsubsection{Linux}
\label{sec:installation-linux}
Use your favorite package manager.

\subsection{Configuration}
\label{sec:configuration}

\begin{enumerate}
  \item basic configuration of the service (ports, memory usage, etc)
  \item advanced: sharding, performance optimization, indexing
  \item advanced: distribution among several network nodes
\end{enumerate}
    
\section{Using MongoDB}
\label{sec:usage}

\subsection{Command Line Interface}
\label{sec:usage-cli}
CRUD operations\\
complex queries

\subsection{Programmatically}
\label{sec:usage-programmatically}
\begin{enumerate}
  \item Using MongoDB as a module of a application is probably the most common usage scenario
  \item Interfaces for a large number of programming languages exist
  \item Some basic operations can be shown in Ruby, Java, Python, JavaScript
\end{enumerate}



\subsection{Graphical Interfaces}
\label{sec:usage-gui}
there are UI based admin tools by 3rd party developers\\
show a small selection of them

\section{Maintaining MongoDB}
\label{sec:maintenance}

\subsection{Monitoring}
\label{sec:maintenance-monitroing}
Can be done manually via the CLI
Monitoring tools for automation are included in the default installation

\subsection{Updating}
\label{sec:maintenance-updating}
TODO



%----------------------------------------------------------------------------
% Closing
%----------------------------------------------------------------------------

\newpage
\section{Conclusion and Outlook}
\label{sec:conclusion}

MongoDB should not be seen as an alternative for conventional RDBS but as a
complementary module in application development.\\
TODO.


%----------------------------------------------------------------------------
% APPENDIX
%----------------------------------------------------------------------------
% Appendix sections need to be within the subappendices environment.
% Use the command \appsection{title} instead of \section to introduce each
% appendix. This will add each appendix to the list of appendices.

% sets the appendix environment and resets the section counters
\newpage \begin{appendices} 
\appendixtocon %adds an 'Appendices' entry to the toc

\appendixpage %prints the title on the page

\subsection*{\listappendixname}
%--------------------------------
% style of the \listofappendices command is defined in header.tex
\listofappendices

% begin appendices on a new page
\newpage

%start environment for subappendices, so that new sections are formatted as
%subsections of appendix
\begin{subappendices}
\renewcommand{\setthesubsection}{\arabic{subsection}:}%

\appsection{Some Source Code}
\label{apx:placeholder}

% close the appendices environment
\end{subappendices}
\end{appendices}
